% =========================================================
% TÊN FILE: main.tex
% CHỨC NĂNG: File chính để biên dịch
% =========================================================

% Khai báo kiểu tài liệu: dùng 'book' cho tài liệu lớn, 12pt chữ, khổ A4
\documentclass[12pt, a4paper, oneside]{book}

% Gọi tất cả các thiết lập từ file preamble
% =========================================================
% PREAMBLE TỐI GIẢN (Để kiểm tra)
% =========================================================
\usepackage[utf8]{inputenc}
\usepackage[T1]{fontenc}
\usepackage[vietnamese]{babel} 

% Goi cac goi toan co ban
\usepackage{amsmath}
\usepackage{amssymb}
\usepackage{amsfonts}

% Goi goi dinh ly co ban
\usepackage{amsthm}

\usepackage{graphicx} % Cho phep chen hinh anh
\usepackage[left=2.5cm, right=2.5cm, top=2.5cm, bottom=2.5cm]{geometry} % Can le

% Thiet lap dinh ly co ban
\theoremstyle{definition}
\newtheorem{dinhnghia}{Định nghĩa}[chapter]

% --- Thông tin tài liệu (cho trang bìa) ---
\title{Ghi chú Đại số Tuyến tính \\ \large Dành cho Sinh viên Nghiên cứu Chuyên sâu}
\author{Tên của Bạn (Your Name)}
\date{\today}

% --- Bắt đầu tài liệu ---
\begin{document}

% --- PHẦN MỞ ĐẦU ---
\frontmatter 
\maketitle   % Tạo trang bìa
\tableofcontents % Tự động tạo mục lục

% --- PHẦN NỘI DUNG CHÍNH ---
\mainmatter 

% GỌI NỘI DUNG CHƯƠNG 1 VÀO:
\chapter{Không gian Vector và Không gian con}
% =========================================================
% CHUONG 1 TỐI GIẢN (Để kiểm tra)
% =========================================================

\chapter{Chương 1: Kiểm tra Biên dịch}

\section{Đây là một mục kiểm tra}

Nội dung này được viết để kiểm tra xem GitHub Actions
có chạy được hay không. Nó không chứa bất kỳ 
ky tu Unicode, emoji, hay dau cach la nao.

\begin{dinhnghia}
Đây là một định nghĩa thử nghiệm.
\end{dinhnghia}

Đây là một công thức toán học cơ bản:
\[
A = \begin{pmatrix} 1 & 0 \\ 0 & 1 \end{pmatrix}
\]

Kết thúc kiểm tra.

% GỌI NỘI DUNG CHƯƠNG 2 VÀO (khi bạn đã viết xong file chuong_02.tex):
%\chapter{Ánh xạ Tuyến tính và Ma trận}
%\input{chapters/chuong_02_AnhXaTuyenTinh.tex}


% --- KẾT THÚC TÀI LIỆU ---
\end{document}