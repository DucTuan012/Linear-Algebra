% =========================================================
% TÊN FILE: preamble.tex
% CHỨC NĂNG: Chứa tất cả các gói (packages) và định nghĩa môi trường
% =========================================================

% --- Cài đặt cơ bản ---
% Bạn sẽ dùng lệnh \documentclass ở file main.tex
\usepackage[utf8]{inputenc}
\usepackage[T1]{fontenc}
\usepackage[vietnamese]{babel} % Hỗ trợ tiếng Việt

% --- Gói Toán học (Rất cần thiết) ---
\usepackage{amsmath}
\usepackage{amssymb}
\usepackage{amsfonts}
\usepackage{mathtools}       % Bổ sung cho amsmath

% --- Gói Định lý & Môi trường Khoa học ---
\usepackage{amsthm}          % Gói để tạo các môi trường Định lý, Định nghĩa
\usepackage{thmtools}        % Tùy biến thêm cho amsthm

% --- Gói Hỗ trợ & Trình bày ---
\usepackage{graphicx}        % Chèn hình ảnh
\usepackage{float}           % Kiểm soát vị trí hình ảnh [H]
\usepackage[left=2.5cm, right=2.5cm, top=2.5cm, bottom=2.5cm]{geometry} % Căn lề

% --- Gói "Cá nhân hóa" (Để tạo style vở ghi) ---
\usepackage[dvipsnames, svgnames]{xcolor} % Gói màu sắc
\usepackage[most]{tcolorbox} % Gói tạo các box ghi chú màu sắc

\usepackage{hyperref}        % Tạo link PDF (PHẢI ĐỂ GẦN CUỐI)
\hypersetup{
    colorlinks=true,
    linkcolor=RoyalBlue,
    urlcolor=RoyalBlue,
    citecolor=ForestGreen,
    pdftitle={Ghi chú Đại số Tuyến tính},
    pdfauthor={Tên của bạn},
}

% =========================================================
% ĐỊNH NGHĨA MÔI TRƯỜNG KHOA HỌC (Định lý, Định nghĩa,...)
% =========================================================
% Đánh số theo Chương (ví dụ: Định lý 1.1, 1.2,...)

\theoremstyle{definition} % Kiểu chữ đứng, in đậm tiêu đề
\newtheorem{dinhnghia}{Định nghĩa}[chapter]
\newtheorem{vidu}{Ví dụ}[chapter]

\theoremstyle{plain}      % Kiểu chữ nghiêng, in đậm tiêu đề
\newtheorem{dinhly}{Định lý}[chapter]
\newtheorem{menhde}{Mệnh đề}[chapter]
\newtheorem{hequa}{Hệ quả}[chapter]

\theoremstyle{remark}     % Kiểu chữ đứng, tiêu đề nghiêng
\newtheorem{chungminh}{Chứng minh}
\newtheorem{ghichu}{Ghi chú}[chapter]


% =========================================================
% ĐỊNH NGHĨA MÔI TRƯỜNG CÁ NHÂN (Dùng tcolorbox)
% =========================================================

% Box "Lưu ý cá nhân" - Phong cách vở ghi
\newtcolorbox{note}{
    colback=LightGoldenrodYellow!20, % Màu nền
    colframe=Goldenrod!80,           % Màu viền
    fonttitle=\bfseries,
    title=💡 Lưu ý cá nhân
}

% Box "Cảnh báo Lỗi sai thường gặp"
\newtcolorbox{warning}{
    colback=Red!10,
    colframe=Red!75!Black,
    fonttitle=\bfseries,
    title=🔥 Lỗi sai thường gặp
}

% Box "Tóm tắt chương"
\newtcolorbox{summary}{
    colback=RoyalBlue!10,
    colframe=RoyalBlue!75!Black,
    fonttitle=\bfseries,
    title=🔑 Tóm tắt cốt lõi
}