% =========================================================
% CHUONG 1 TỐI GIẢN (Để kiểm tra)
% =========================================================

\chapter{Ma trận}

\section{Các khái niệm cơ bản về ma trận}
\textbf{Định nghĩa Ma trận (Definition of a Matrix)}
Cho $K$ là một **trường** (field) bất kỳ. Thông thường, $K$ có thể là trường số thực $\mathbb{R}$ hoặc trường số phức $\mathbb{C}$.

Một **ma trận** (matrix) $A$ cấp (hoặc kích thước) $m \times n$ trên trường $K$ là một bảng hình chữ nhật (rectangular array) gồm $m \cdot n$ phần tử (elements) của $K$, được sắp xếp thành $m$ hàng (rows) và $n$ cột (columns).

Ma trận $A$ được biểu diễn tổng quát dưới dạng:

$$
A =
\begin{pmatrix}
a_{11} & a_{12} & \cdots & a_{1n} \\
a_{21} & a_{22} & \cdots & a_{2n} \\
\vdots & \vdots & \ddots & \vdots \\
a_{m1} & a_{m2} & \cdots & a_{mn}
\end{pmatrix}
$$

**Trong đó:**
* $m$ và $n$ là các số nguyên dương ($m, n \in \mathbb{N}^*$). $m$ được gọi là **số hàng** và $n$ được gọi là **số cột**.
* $a_{ij} \in K$ là phần tử của ma trận nằm tại vị trí giao của **hàng thứ $i$** (với $1 \le i \le m$) và **cột thứ $j$** (với $1 \le j \le n$). Chỉ số $i$ là chỉ số hàng và chỉ số $j$ là chỉ số cột.

**Ký hiệu (Notation):**
1.  Ma trận $A$ thường được ký hiệu rút gọn là $A = [a_{ij}]_{m \times n}$, hoặc $A = (a_{ij})_{m \times n}$. Nếu kích thước $m \times n$ đã được xác định rõ trong ngữ cảnh, có thể viết đơn giản là $A = [a_{ij}]$.
2.  Tập hợp tất cả các ma trận cấp $m \times n$ với các phần tử thuộc trường $K$ được ký hiệu là $M_{m \times n}(K)$. Do đó, ta có thể viết $A \in M_{m \times n}(K)$.

**Các trường hợp đặc biệt:**
* **Ma trận vuông (Square matrix):** Khi $m = n$, $A$ được gọi là ma trận vuông cấp $n$. Tập hợp các ma trận vuông cấp $n$ trên $K$ được ký hiệu là $M_n(K)$. Các phần tử $a_{11}, a_{22}, \dots, a_{nn}$ (tức là $a_{ii}$) tạo thành **đường chéo chính** (main diagonal) của ma trận.
* **Vector cột (Column vector):** Khi $n = 1$ (ma trận cấp $m \times 1$), $A$ được gọi là một vector cột.
* **Vector hàng (Row vector):** Khi $m = 1$ (ma trận cấp $1 \times n$), $A$ được gọi là một vector hàng.


\section{Định nghĩa chính tắc (Công thức Leibniz)}

Đây là định nghĩa tổng quát và cơ bản nhất của định thức.

Cho $A$ là một ma trận vuông cấp $n \times n$, với các phần tử được ký hiệu là $a_{ij}$ (phần tử ở hàng $i$, cột $j$).
$$ A = \begin{pmatrix}
a_{11} & a_{12} & \cdots & a_{1n} \\
a_{21} & a_{22} & \cdots & a_{2n} \\
\vdots & \vdots & \ddots & \vdots \\
a_{n1} & a_{n2} & \cdots & a_{nn}
\end{pmatrix} $$
Định thức của $A$, ký hiệu là $\det(A)$ hoặc $|A|$, được định nghĩa bằng \textbf{công thức Leibniz} như sau:

$$ \det(A) = \sum_{\sigma \in S_n} \text{sgn}(\sigma) \prod_{i=1}^{n} a_{i, \sigma(i)} $$

\subsection*{Giải thích các thành phần}
\begin{itemize}
    \item \textbf{$S_n$}: Là tập hợp tất cả các \textbf{hoán vị} $\sigma$ của $n$ số nguyên $\{1, 2, \dots, n\}$. Một hoán vị $\sigma$ là một cách sắp xếp lại thứ tự của các số này. Ví dụ, nếu $n=3$, một hoán vị có thể là $\sigma = (2, 3, 1)$, nghĩa là $\sigma(1)=2$, $\sigma(2)=3$, và $\sigma(3)=1$.
    
    \item \textbf{$\text{sgn}(\sigma)$}: Là \textbf{dấu của hoán vị} $\sigma$ (signum).
    \begin{itemize}
        \item $\text{sgn}(\sigma) = +1$ nếu $\sigma$ là một \textbf{hoán vị chẵn} (có thể thu được bằng một số chẵn các phép tráo đổi 2 phần tử).
        \item $\text{sgn}(\sigma) = -1$ nếu $\sigma$ là một \textbf{hoán vị lẻ} (có thể thu được bằng một số lẻ các phép tráo đổi 2 phần tử).
    \end{itemize}
    
    \item \textbf{$\prod_{i=1}^{n} a_{i, \sigma(i)}$}: Là tích của $n$ phần tử của ma trận, được lấy sao cho: từ hàng 1 ta lấy phần tử ở cột $\sigma(1)$, từ hàng 2 ta lấy phần tử ở cột $\sigma(2)$, và cứ thế đến hàng $n$ ta lấy phần tử ở cột $\sigma(n)$. Điều này đảm bảo mỗi hàng và mỗi cột chỉ được đại diện đúng một lần trong mỗi tích.
\end{itemize}

Nói một cách dễ hiểu, định thức là tổng của tất cả $n!$ (n giai thừa) tích có thể có, mỗi tích được tạo thành từ $n$ phần tử (mỗi hàng và mỗi cột chỉ lấy một phần tử), với dấu cộng hoặc trừ được quyết định bởi tính chẵn/lẻ của hoán vị các chỉ số cột.

\section{Định nghĩa đệ quy (Khai triển Laplace)}

Đây là một định nghĩa mang tính xây dựng và thường được dùng để tính toán định thức trong thực tế.

\subsection*{Trường hợp cơ sở}
Nếu $A$ là ma trận $1 \times 1$, $A = [a_{11}]$, thì $\det(A) = a_{11}$.

\subsection*{Bước đệ quy (cho $n > 1$)}
Để định nghĩa định thức của ma trận $A$ cấp $n \times n$, ta cần hai khái niệm:

\begin{enumerate}
    \item \textbf{Ma trận con (Minor)}: \textbf{Ma trận con} $A_{ij}$ (đôi khi ký hiệu là $M_{ij}$) là ma trận cấp $(n-1) \times (n-1)$ thu được bằng cách \textbf{xóa hàng $i$ và cột $j$} của ma trận $A$.
    
    \item \textbf{Phần bù đại số (Cofactor)}: \textbf{Phần bù đại số} $C_{ij}$ của phần tử $a_{ij}$ được định nghĩa là:
    $$ C_{ij} = (-1)^{i+j} \det(A_{ij}) $$
    Dấu $(-1)^{i+j}$ tạo ra một "bàn cờ" dấu:
    $$ \begin{pmatrix}
    + & - & + & \cdots \\
    - & + & - & \cdots \\
    + & - & + & \cdots \\
    \vdots & \vdots & \vdots & \ddots
    \end{pmatrix} $$
\end{enumerate}

Với các khái niệm trên, định thức của $A$ có thể được tính bằng cách \textbf{khai triển theo một hàng $i$ bất kỳ} (từ $1$ đến $n$):
$$ \det(A) = \sum_{j=1}^{n} a_{ij} C_{ij} = a_{i1}C_{i1} + a_{i2}C_{i2} + \dots + a_{in}C_{in} $$

Hoặc, \textbf{khai triển theo một cột $j$ bất kỳ} (từ $1$ đến $n$):
$$ \det(A) = \sum_{i=1}^{n} a_{ij} C_{ij} = a_{1j}C_{1j} + a_{2j}C_{2j} + \dots + a_{nj}C_{nj} $$

Giá trị của định thức là duy nhất, không phụ thuộc vào hàng hay cột nào được chọn để khai triển. Định nghĩa này mang tính đệ quy vì $\det(A)$ (cấp $n$) được định nghĩa thông qua các $\det(A_{ij})$ (cấp $n-1$).

\end{document}