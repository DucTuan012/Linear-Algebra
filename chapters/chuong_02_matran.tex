Định nghĩa Ma trận (Definition of a Matrix)

Cho $K$ là một **trường** (field) bất kỳ. Thông thường, $K$ có thể là trường số thực $\mathbb{R}$ hoặc trường số phức $\mathbb{C}$.

Một **ma trận** (matrix) $A$ cấp (hoặc kích thước) $m \times n$ trên trường $K$ là một bảng hình chữ nhật (rectangular array) gồm $m \cdot n$ phần tử (elements) của $K$, được sắp xếp thành $m$ hàng (rows) và $n$ cột (columns).

Ma trận $A$ được biểu diễn tổng quát dưới dạng:

$$
A =
\begin{pmatrix}
a_{11} & a_{12} & \cdots & a_{1n} \\
a_{21} & a_{22} & \cdots & a_{2n} \\
\vdots & \vdots & \ddots & \vdots \\
a_{m1} & a_{m2} & \cdots & a_{mn}
\end{pmatrix}
$$

**Trong đó:**
* $m$ và $n$ là các số nguyên dương ($m, n \in \mathbb{N}^*$). $m$ được gọi là **số hàng** và $n$ được gọi là **số cột**.
* $a_{ij} \in K$ là phần tử của ma trận nằm tại vị trí giao của **hàng thứ $i$** (với $1 \le i \le m$) và **cột thứ $j$** (với $1 \le j \le n$). Chỉ số $i$ là chỉ số hàng và chỉ số $j$ là chỉ số cột.

**Ký hiệu (Notation):**
1.  Ma trận $A$ thường được ký hiệu rút gọn là $A = [a_{ij}]_{m \times n}$, hoặc $A = (a_{ij})_{m \times n}$. Nếu kích thước $m \times n$ đã được xác định rõ trong ngữ cảnh, có thể viết đơn giản là $A = [a_{ij}]$.
2.  Tập hợp tất cả các ma trận cấp $m \times n$ với các phần tử thuộc trường $K$ được ký hiệu là $M_{m \times n}(K)$. Do đó, ta có thể viết $A \in M_{m \times n}(K)$.

**Các trường hợp đặc biệt:**
* **Ma trận vuông (Square matrix):** Khi $m = n$, $A$ được gọi là ma trận vuông cấp $n$. Tập hợp các ma trận vuông cấp $n$ trên $K$ được ký hiệu là $M_n(K)$. Các phần tử $a_{11}, a_{22}, \dots, a_{nn}$ (tức là $a_{ii}$) tạo thành **đường chéo chính** (main diagonal) của ma trận.
* **Vector cột (Column vector):** Khi $n = 1$ (ma trận cấp $m \times 1$), $A$ được gọi là một vector cột.
* **Vector hàng (Row vector):** Khi $m = 1$ (ma trận cấp $1 \times n$), $A$ được gọi là một vector hàng.
