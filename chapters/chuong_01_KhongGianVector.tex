% =========================================================
% TÊN FILE: chapters/chuong_01_KhongGianVector.tex
% NỘI DUNG: Định nghĩa Định thức qua Hoán vị
% =========================================================

\chapter{Định thức của Ma trận Vuông}
\label{chuong:dinhthuc}

\section{Định nghĩa bằng Hoán vị}

Đây là định nghĩa chính tắc và tổng quát nhất của định thức, thường được sử dụng trong các chứng minh lý thuyết.

\begin{dinhnghia}[Định thức (Định nghĩa Leibniz)]
Cho một ma trận vuông $A$ cấp $n \times n$, với $A = [a_{ij}]$. Định thức của $A$, ký hiệu là $\det(A)$ hoặc $|A|$, được định nghĩa như sau:

\begin{equation}
\label{eq:det_leibniz}
\det(A) = \sum_{\sigma \in S_n} \text{sgn}(\sigma) \prod_{i=1}^n a_{i, \sigma(i)}
\end{equation}

Trong đó:
\begin{itemize}
    \item $S_n$ là tập hợp tất cả các hoán vị của $n$ phần tử $\{1, 2, \dots, n\}$. Tập hợp này có $n!$ (n giai thừa) phần tử.
    \item $\sigma$ là một hoán vị cụ thể trong $S_n$. $\sigma(i)$ là giá trị của phần tử thứ $i$ sau khi hoán vị.
    \item $\text{sgn}(\sigma)$ là \textbf{dấu (sign)} của hoán vị $\sigma$. Nó bằng $+1$ nếu $\sigma$ là hoán vị chẵn (có thể biểu diễn bằng một số chẵn lần đổi chỗ), và bằng $-1$ nếu $\sigma$ là hoán vị lẻ.
\end{itemize}
\end{dinhnghia}

\begin{note}
Nói một cách dễ hiểu: Định thức là tổng của $n!$ số hạng. Mỗi số hạng là tích của $n$ phần tử của ma trận (mỗi hàng và mỗi cột chỉ được lấy đúng một lần), nhân với dấu $(+1)$ hoặc $(-1)$ tùy thuộc vào hoán vị của các chỉ số cột.
\end{note}

\section{Ví dụ Minh họa Định nghĩa}

\begin{vidu}[Trường hợp $n=2$]
Cho ma trận $A = \begin{pmatrix} a_{11} & a_{12} \\ a_{21} & a_{22} \end{pmatrix}$.
Tập hoán vị $S_2$ có $2! = 2$ phần tử:
\begin{enumerate}
    \item $\sigma_1 = (1, 2)$ (hoán vị đồng nhất): chẵn, $\text{sgn}(\sigma_1) = +1$.
    \item $\sigma_2 = (2, 1)$ (đổi chỗ 1 và 2): lẻ, $\text{sgn}(\sigma_2) = -1$.
\end{enumerate}
Áp dụng công thức (\ref{eq:det_leibniz}):
\begin{align*}
\det(A) &= \text{sgn}(\sigma_1) a_{1, \sigma_1(1)} a_{2, \sigma_1(2)} + \text{sgn}(\sigma_2) a_{1, \sigma_2(1)} a_{2, \sigma_2(2)} \\
&= (+1) a_{11} a_{22} + (-1) a_{12} a_{21} \\
&= a_{11}a_{22} - a_{12}a_{21}
\end{align*}
Đây chính là công thức định thức cấp 2 quen thuộc.
\end{vidu}

\begin{vidu}[Trường hợp $n=3$ (Quy tắc Sarrus)]
Tập $S_3$ có $3! = 6$ hoán vị:

\textbf{Các hoán vị chẵn ($\text{sgn} = +1$):}
\begin{itemize}
    \item $\sigma = (1, 2, 3)$ (đồng nhất) $\to +a_{11}a_{22}a_{33}$
    \item $\sigma = (2, 3, 1)$ (dịch vòng) $\to +a_{12}a_{23}a_{31}$
    \item $\sigma = (3, 1, 2)$ (dịch vòng) $\to +a_{13}a_{21}a_{32}$
\end{itemize}

\textbf{Các hoán vị lẻ ($\text{sgn} = -1$):}
\begin{itemize}
    \item $\sigma = (1, 3, 2)$ (đổi chỗ 2, 3) $\to -a_{11}a_{23}a_{32}$
    \item $\sigma = (2, 1, 3)$ (đổi chỗ 1, 2) $\to -a_{12}a_{21}a_{33}$
    \item $\sigma = (3, 2, 1)$ (đổi chỗ 1, 3) $\to -a_{13}a_{22}a_{31}$
\end{itemize}

Tổng của 6 số hạng này chính là công thức khai triển Sarrus cho định thức cấp 3.
\end{vidu}